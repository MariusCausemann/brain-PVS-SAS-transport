\begin{table}
\begin{center}
  \begin{tabular}{ll|llllll|ll}
    \toprule
    & Description & $R$ & $\xi_{\rm PVS-SAS}$, $\xi_{\rm PVS-PAR}$ & $\beta_{\rm efflux}$ & $D$ & $\hat{u}$ & $\mathbf{u}_{\rm SAS, PAR}$ & $g_{\rm influx}$ & $c_0$ \\
    \midrule
    A & Baseline  & $R_2 = 2 R_1$ & $\xi$, $\tilde \xi$\cite{koch2023estimates} &  $10^{-4}$ mm$^2$/s\cite{hornkjol2022csf} & $D^{\rm gad}$\cite{sykova2008diffusion, valnes2020apparent}  & $\hat{u} = \hat{u}_{\rm prod}$ & $\mathbf{u}_{\rm SAS} = \mathbf{u}_{\rm prod}$, $\mathbf{u}_{\rm PAR} = 0$ & $> 0$ & 0 \\
    B & PVS "sheaths" & $R_2 = 2 R_1$ & $0, 0.5 \xi, \xi, 2 \xi, 1000 \xi \approx \infty$, --\cite{koch2023estimates} & -- & --  &  --  & --  & -- & -- \\
    C & Larger PVS & $R_2 = N R_1$ & -- & -- & --  &  --  & -- & -- & -- \\
    D & PVS I & $R_2 = 2 R_1$ & $\xi$, -- & -- & --  &  $\hat{u} = \hat{u}_{\rm prod} + \uparrow$  & -- & -- & -- \\
    E & PVS II & $R_2 = 2 R_1$ & $\xi$, -- & -- & --  &  $\hat{u} = \hat{u}_{\rm prod} + \uparrow\uparrow$  & -- & -- & -- \\
    F & PVS III & $R_2 = 2 R_1$ & $\xi$, -- & -- & $10 D_{\rm PVS}$ &  $\hat{u} = \hat{u}_{\rm prod}$  & -- & -- & -- \\
    G & Glymphatics & $R_2 = 2 R_1$ & $\xi$, -- & -- & \cite{sykova2008diffusion, valnes2020apparent} &  $\hat{u} = \hat{u}_{\rm prod} + \uparrow$  & $\mathbf{u}_{\rm PAR}$ > 0 & -- & -- \\
    H & Sleep &  &  & &  &  &  & & \\
    I & Pathology I &  &  & &  &  &  & & \\
    J & Pathology II &  &  & &  &  &  & & \\
    \midrule
    X & Clearance & $R_2 = 2 R_1$ & $50\%$, $\xi$ & -- & $D^{\rm gad, \tau, A\beta}$  &  $\hat{u} = \hat{u}_{\rm prod}$  & -- & $0$ & $1$ \\
    Y & (PVS) & $R_2 = 2 R_1$ & $50\%$, $\xi$ & -- & $D^{\rm gad, \tau, A\beta}$  &  $\hat{u} = \hat{u}_{\rm prod} + \uparrow$  & -- & -- & -- \\
    Z & (Glymphatic) & $R_2 = 2 R_1$ & $50\%$, $\xi$ & -- & $D^{\rm gad, \tau, A\beta}$  &  $\hat{u} = \hat{u}_{\rm prod} + \uparrow$  & $\mathbf{u}_{\rm PAR}$ > 0 & -- & -- \\
    \bottomrule
    \end{tabular}
    \end{center}
\caption{Overview of models (-- denotes as immediately above). Model A represents a baseline scenario with semi-permeable barriers between the PVS and SAS ($\xi_{\rm PVS-SAS} \approx 50\%$, \mer{we try to estimate this value by a bit of trial-and-error from the observations of timings of PVS-SAS and SAS from~\cite{eide2024functional}})~\cite{bedussi2017paravascular} and astrocytic endfeet forming barriers within the parenchyma~\cite{koch2023estimates}. $R_1$ and $R_2$ denotes the inner and outer radius of the PVS, with PVS width comparable to vascular diameter ($R_2 - R_1 \approx 2 R_1$)\cite{mestre2018flow} as a baseline. The extracranial solute efflux permeability $\beta_{\rm efflux}$ is uniformly distributed over the outer SAS boundary with a reasonable rate as baseline\cite{hornkjol2022csf, eide2021clinical, ringstad2024glymphatic}. At baseline, the effective diffusion coefficients $D^{\rm x}_{\rm SAS} = D^{\rm x}_{\rm PVS}$ of the SAS and PVS represent the free diffusion coefficient of Gadubutrol (${\rm x} = {\rm gad}$) in CSF (water), and $D^{\rm x}_{\rm PAR}$ the effective diffusion coefficient of human cortical tissue, all at body temperature. At baseline, we consider net flow due to CSF-production in the SAS (and PVS): $\mathbf{u}_{\rm SAS} > 0$ (and $\hat{u} > 0$), but no additional effects from perivascular pulsatility nor bulk flow in the parenchyma $\mathbf{u}_{\rm PAR} = 0$. Due to the timescale for human intracranial tracer transport (minutes to hours) compared to typical (cardiac, respiratory, slow vasomotion) human intracranial pulsatility (0.1 -- 1 Hz), we consider net (constant-in-time) flow fields only in subsequent model variations. Model B represents a PVS-SAS interface configuration with minimal or more permeable structural barriers~\cite{eide2024functional}, while Model C represents enlarged PVS. Models D, E, and F represent three perivascular pathway scenarios with more rapid flow in the PVS induced by vasomotion (PVS I)~\cite{gjerde2023directional} or a net fluid source in the PVS (1D Stokes flow)/net pressure difference between PVS inlets and outlets (PVS II), or with enhanced effective diffusion due to mixing (PVS III)~\cite{hornkjol2022csf, troyetsky2021dispersion, vinje2023human}, respectively. Model G attempts to represent a glymphatic--like convective flow within the parenchyma. \mer{Ideas could be:  (a) network induces arterial and venous sources in a Darcy porous media flow model (reflects the glymphatic hypothesis), (b) Croci-style \cite{croci2019uncertainty}, (c) \cite{vinje2023human}} Model H represents physiological alterations associated with sleep (lower CSF production, enhanced diffusion within the parenchyma, enhanced PVS pulsatility), while Models I--J represents variations associated with pathologies (e.g.~dementia, hypertension, angiopathies, CSF-disorders and/or others). Models X--Z represent version of the aforementioned scenarios but for studying metabolite clearance (instead of solute influx).}
\label{tab:scenarios}
\end{table}
