%% \documentclass{article}
%% \usepackage{a4wide}
%% \usepackage{graphicx}
%% \usepackage{subcaption}
%% \usepackage{tikz}

%% \usetikzlibrary{quotes}
%% \usetikzlibrary{arrows,decorations.pathmorphing,backgrounds,positioning,fit,petri}

%% %\begin{document}
%% \captionsetup[subfigure]{justification=justified,singlelinecheck=false}

\tikzset{
  vertex/.style={circle, fill=black!15, font=\tiny, inner sep=2pt},
  blank/.style={circle, fill=black!05, font=\tiny},
}
\begin{figure}
\centering
  \begin{subfigure}{0.19\textwidth}
  \begin{tikzpicture}
  [scale=0.7]
  \node[vertex, fill=red!50] (n3) at (-1, 0) {}; 
  \node[vertex, fill=orange!50] (n0) at (1, 0) {}; 
  \node[vertex] (n2) at (-0.5, 1) {}; 
  \node[vertex] (n1) at (0.5, 1) {}; 

  \node[vertex] (n4) at (-1, 2) {}; 
  \node[vertex] (n5) at (-1.5, 3) {}; 
  \node[vertex] (n6) at (-0.5, 3) {}; 
  \node[vertex] (n7) at (-1.0, 4) {}; 

  \node[vertex] (n8) at (1.0, 2) {}; 
  \node[vertex] (n9) at (1.0, 3) {}; 
  \node[vertex] (n10) at (0.5, 4) {}; 
  \node[vertex] (n11) at (1.5, 4) {}; 
  \node[vertex] (n12) at (2.0, 5) {}; 
  \node[vertex] (n13) at (0.5, 2.5) {}; 
  \node[vertex] (n14) at (0, 5) {}; 
  \node[vertex] (n15) at (2, 6) {}; 

  \draw[ ultra thick ] (n3) -- (n2);% node[right]{$ \bf x$};
  \draw[ ultra thick ] (n2) -- (n1);
  \draw[ ultra thick ] (n0) -- (n1);

  \draw[ very thick ] (n2) -- (n4);
  \draw[ very thick ] (n4) -- (n5);
  \draw[ thick ] (n4) -- (n6);
  \draw[ thick ] (n6) -- (n7);

  \draw[ very thick ] (n1) -- (n8);
  \draw[ very thick ] (n8) -- (n9);

  \draw[ thick ] (n9) -- (n10);
  \draw[ thick ] (n10) -- (n14);

  \draw[ thick] (n9) -- (n13);

  \draw[ very thick ] (n9) -- (n11);
  \draw[ very thick ] (n11) -- (n12);
  \draw[ very thick ] (n12) -- (n15);

\end{tikzpicture}
\caption*{\bf A}
  \end{subfigure}
\begin{subfigure}{0.19\textwidth}
\begin{tikzpicture}
  [scale=0.7,
  ]

  \node[vertex,fill=red!50] (n3) at (-1, 0) {}; 
  \node[vertex,fill=orange!50] (n0) at (1, 0) {}; 
  \node[vertex,fill=red!50] (n2) at (-0.5, 1) {}; 
  \node[vertex,fill=orange!50] (n1) at (0.5, 1) {}; 

  \node[vertex,fill=red!50] (n4) at (-1, 2) {}; 
  \node[vertex,fill=red!50] (n5) at (-1.5, 3) {}; 
  \node[vertex,fill=red!50] (n6) at (-0.5, 3) {}; 
  \node[vertex,fill=red!50] (n7) at (-1.0, 4) {}; 

  \node[vertex,fill=orange!50] (n8) at (1.0, 2) {}; 
  \node[vertex,fill=orange!50] (n9) at (1.0, 3) {}; 
  \node[vertex,fill=orange!50] (n10) at (0.5, 4) {}; 
  \node[vertex,fill=orange!50] (n11) at (1.5, 4) {}; 
  \node[vertex,fill=orange!50] (n12) at (2.0, 5) {}; 
  \node[vertex,fill=orange!50] (n13) at (0.5, 2.5) {}; 
  \node[vertex,fill=orange!50] (n14) at (0, 5) {}; 
  \node[vertex,fill=orange!50] (n15) at (2, 6) {}; 

  \draw[ ultra thick ] (n3) -- (n2);% node[right]{$ \bf x$};

  \draw[ ultra thick ] (n0) -- (n1);

  \draw[ very thick ] (n2) -- (n4);
  \draw[ very thick ] (n4) -- (n5);
  \draw[ thick ] (n4) -- (n6);
  \draw[ thick ] (n6) -- (n7);

  \draw[ very thick ] (n1) -- (n8);
  \draw[ very thick ] (n8) -- (n9);

  \draw[ thick ] (n9) -- (n10);
  \draw[ thick ] (n10) -- (n14);

  \draw[ thick, dotted ] (n9) -- (n13);

  \draw[ very thick ] (n9) -- (n11);
  \draw[ very thick ] (n11) -- (n12);
  \draw[ very thick ] (n12) -- (n15);
\end{tikzpicture}
\caption*{\bf B}
\end{subfigure}
  \begin{subfigure}{0.19\textwidth}
\begin{tikzpicture}
  [scale=0.7,
  ]
  \node[vertex,fill=red!50] (n3) at (-1, 0) {}; 
  \node[vertex,fill=red!50] (n4) at (-1, 2) {}; 
  \node[vertex,fill=red!50] (n5) at (-1.5, 3) {}; 
  \node[vertex,fill=red!50] (n7) at (-1.0, 4) {}; 

  \draw[-stealth, ultra thick ] (n3) to  [bend right=20] (n4) ;
  \draw[-stealth, very thick ] (n4) -- (n5);
  \draw[-stealth, thick ] (n4) to [bend right=20] (n7);

  \node[vertex, fill=orange!50] (n0) at (1, 0) {}; 
  \node[vertex, fill=orange!50] (n9) at (1.0, 3) {}; 
  \node[vertex, fill=orange!50] (n14) at (0, 5) {}; 
  \node[vertex, fill=orange!50] (n15) at (2, 6) {}; 

  \draw[-stealth, ultra thick ] (n0) to [bend left=10] (n9);
  \draw[-stealth, thick ] (n9) -- (n14);
  \draw[-stealth, very thick ] (n9) to [bend right=10] (n15);
\end{tikzpicture}
\caption*{\bf C}
  \end{subfigure}
  \begin{subfigure}{0.19\textwidth}
\begin{tikzpicture}
  [scale=0.7,
  ]
  \node[vertex, fill=red!50] (n3) at (-1, 0) {}; 
  \node[vertex] (n4) at (-1, 2) {}; 
  \node[vertex] (n5) at (-1.5, 3) {}; 
  \node[vertex] (n7) at (-1.0, 4) {}; 


% alpha = 1.0
%Running simple schematic test case via estimate_net_flow
%<Q'>_n (mm^3/s) =  [0.00707734 0.00634445 0.0007329 ]
%<u'>_n (mm/s) =  [0.00312887 0.00631094 0.0029161 ]

  \draw[-stealth, ultra thick, blue!40] (n3) to  [bend right=20] (n4) ;
  \draw[-stealth, very thick, blue!30] (n4) -- (n5);
  \draw[-stealth, thick, blue!10] (n4) to [bend right=20] (n7);

  \node[vertex, fill=orange!50] (n0) at (1, 0) {}; 
  \node[vertex] (n9) at (1.0, 3) {}; 
  \node[vertex] (n14) at (0, 5) {}; 
  \node[vertex] (n15) at (2, 6) {}; 

% alpha = 2.0
%Running simple schematic test case via estimate_net_flow
%<Q'>_n (mm^3/s) =  [0.01923927 0.01708518 0.00215408]
%<u'>_n (mm/s) =  [0.00850562 0.01699495 0.00857082]

  \draw[-stealth, ultra thick, blue!80] (n0) to [bend left=10] (n9);
  \draw[-stealth, thick, blue!20] (n9) -- (n14);
  \draw[-stealth, very thick, blue!70] (n9) to [bend right=10] (n15);
\end{tikzpicture}
\caption*{\bf D}
  \end{subfigure}
\begin{subfigure}{0.19\textwidth}
  \begin{tikzpicture}
  [scale=0.7]
  \node[vertex, fill=red!50] (n3) at (-1, 0) {}; 
  \node[vertex, fill=orange!50] (n0) at (1, 0) {}; 
  \node[vertex] (n2) at (-0.5, 1) {}; 
  \node[vertex] (n1) at (0.5, 1) {}; 

  \node[vertex] (n4) at (-1, 2) {}; 
  \node[vertex] (n5) at (-1.5, 3) {}; 
  \node[vertex] (n6) at (-0.5, 3) {}; 
  \node[vertex] (n7) at (-1.0, 4) {}; 

  \node[vertex] (n8) at (1.0, 2) {}; 
  \node[vertex] (n9) at (1.0, 3) {}; 
  \node[vertex] (n10) at (0.5, 4) {}; 
  \node[vertex] (n11) at (1.5, 4) {}; 
  \node[vertex] (n12) at (2.0, 5) {}; 
  \node[vertex] (n13) at (0.5, 2.5) {}; 
  \node[vertex] (n14) at (0, 5) {}; 
  \node[vertex] (n15) at (2, 6) {}; 

  \draw[-stealth, ultra thick, blue!40] (n3) -- (n2);% node[right]{$ \bf x$};
  \draw[-, ultra thick, black!10 ] (n2) -- (n1);
  \draw[-stealth, ultra thick, blue!80] (n0) -- (n1);

  \draw[-stealth, very thick, blue!40 ] (n2) -- (n4);
  \draw[-stealth, very thick, blue!30 ] (n4) -- (n5);
  \draw[-stealth, thick, blue!10 ] (n4) -- (n6);
  \draw[-stealth, thick, blue!10 ] (n6) -- (n7);

  \draw[-stealth, very thick, blue!80 ] (n1) -- (n8);
  \draw[-stealth, very thick, blue!80 ] (n8) -- (n9);

  \draw[-stealth, thick, blue!20 ] (n9) -- (n10);
  \draw[-stealth, thick, blue!20 ] (n10) -- (n14);

  \draw[ thick, black!10] (n9) -- (n13);

  \draw[-stealth, very thick, blue!70 ] (n9) -- (n11);
  \draw[-stealth, very thick, blue!70 ] (n11) -- (n12);
  \draw[-stealth, very thick, blue!70 ] (n12) -- (n15);

\end{tikzpicture}
\caption*{\bf E}
  \end{subfigure}
\caption{Schematic illustrating network representations for the
    estimation of perivascular flow induced by vascular wall
    motion. (\textbf{A}) For a vascular network with vessels
    represented by edges of varying radii and length and connected at
    nodes with one or more supply nodes (here two, marked in red and
    orange), we compute one subnetwork for each supply node by
    proximity (\textbf{B}). Each subnetwork is reduced to a minimal,
    bifurcating and directed tree (\textbf{C}) which is then used to
    compute $\langle Q' \rangle$.}
\end{figure}
%\end{document}
