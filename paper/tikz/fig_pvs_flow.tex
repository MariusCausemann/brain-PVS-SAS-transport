%% \documentclass{article}
%% \usepackage{a4wide}
%% \usepackage{graphicx}
%% \usepackage{subcaption}
%% \usepackage{tikz}

%% \usetikzlibrary{quotes}
%% \usetikzlibrary{arrows,decorations.pathmorphing,backgrounds,positioning,fit,petri}

%% %\begin{document}
%% \captionsetup[subfigure]{justification=justified,singlelinecheck=false}

\tikzset{
  vertex/.style={circle, fill=black!15, font=\tiny},
  blank/.style={circle, fill=black!05, font=\tiny},
}
\begin{figure}
\centering
  \begin{subfigure}{0.24\textwidth}
  \begin{tikzpicture}
  [scale=0.7]
  \node[vertex, fill=red!50] (n3) at (-1, 0) {}; 
  \node[vertex, fill=orange!50] (n0) at (1, 0) {}; 
  \node[vertex] (n2) at (-0.5, 1) {}; 
  \node[vertex] (n1) at (0.5, 1) {}; 

  \node[vertex] (n4) at (-1, 2) {}; 
  \node[vertex] (n5) at (-1.5, 3) {}; 
  \node[vertex] (n6) at (-0.5, 3) {}; 
  \node[vertex] (n7) at (-1.0, 4) {}; 

  \node[vertex] (n8) at (1.0, 2) {}; 
  \node[vertex] (n9) at (1.0, 3) {}; 
  \node[vertex] (n10) at (0.5, 4) {}; 
  \node[vertex] (n11) at (1.5, 4) {}; 
  \node[vertex] (n12) at (2.0, 5) {}; 
  \node[vertex] (n13) at (0.5, 2.5) {}; 
  \node[vertex] (n14) at (0, 5) {}; 
  \node[vertex] (n15) at (2, 6) {}; 

  \draw[ ultra thick ] (n3) -- (n2);% node[right]{$ \bf x$};
  \draw[ ultra thick ] (n2) -- (n1);
  \draw[ ultra thick ] (n0) -- (n1);

  \draw[ very thick ] (n2) -- (n4);
  \draw[ very thick ] (n4) -- (n5);
  \draw[ thick ] (n4) -- (n6);
  \draw[ thick ] (n6) -- (n7);

  \draw[ very thick ] (n1) -- (n8);
  \draw[ very thick ] (n8) -- (n9);

  \draw[ thick ] (n9) -- (n10);
  \draw[ thick ] (n10) -- (n14);

  \draw[ thick] (n9) -- (n13);

  \draw[ very thick ] (n9) -- (n11);
  \draw[ very thick ] (n11) -- (n12);
  \draw[ very thick ] (n12) -- (n15);

\end{tikzpicture}
\caption*{\bf A}
  \end{subfigure}
\begin{subfigure}{0.24\textwidth}
\begin{tikzpicture}
  [scale=0.7,
  ]

  \node[vertex,fill=red!50] (n3) at (-1, 0) {}; 
  \node[vertex,fill=orange!50] (n0) at (1, 0) {}; 
  \node[vertex,fill=red!50] (n2) at (-0.5, 1) {}; 
  \node[vertex,fill=orange!50] (n1) at (0.5, 1) {}; 

  \node[vertex,fill=red!50] (n4) at (-1, 2) {}; 
  \node[vertex,fill=red!50] (n5) at (-1.5, 3) {}; 
  \node[vertex,fill=red!50] (n6) at (-0.5, 3) {}; 
  \node[vertex,fill=red!50] (n7) at (-1.0, 4) {}; 

  \node[vertex,fill=orange!50] (n8) at (1.0, 2) {}; 
  \node[vertex,fill=orange!50] (n9) at (1.0, 3) {}; 
  \node[vertex,fill=orange!50] (n10) at (0.5, 4) {}; 
  \node[vertex,fill=orange!50] (n11) at (1.5, 4) {}; 
  \node[vertex,fill=orange!50] (n12) at (2.0, 5) {}; 
  \node[vertex,fill=orange!50] (n13) at (0.5, 2.5) {}; 
  \node[vertex,fill=orange!50] (n14) at (0, 5) {}; 
  \node[vertex,fill=orange!50] (n15) at (2, 6) {}; 

  \draw[ ultra thick ] (n3) -- (n2);% node[right]{$ \bf x$};

  \draw[ ultra thick ] (n0) -- (n1);

  \draw[ very thick ] (n2) -- (n4);
  \draw[ very thick ] (n4) -- (n5);
  \draw[ thick ] (n4) -- (n6);
  \draw[ thick ] (n6) -- (n7);

  \draw[ very thick ] (n1) -- (n8);
  \draw[ very thick ] (n8) -- (n9);

  \draw[ thick ] (n9) -- (n10);
  \draw[ thick ] (n10) -- (n14);

  \draw[ thick, dotted ] (n9) -- (n13);

  \draw[ very thick ] (n9) -- (n11);
  \draw[ very thick ] (n11) -- (n12);
  \draw[ very thick ] (n12) -- (n15);
\end{tikzpicture}
\caption*{\bf B}
\end{subfigure}
  \begin{subfigure}{0.24\textwidth}
\begin{tikzpicture}
  [scale=0.7,
  ]
  \node[vertex,fill=red!50] (n3) at (-1, 0) {}; 
  \node[vertex,fill=red!50] (n4) at (-1, 2) {}; 
  \node[vertex,fill=red!50] (n5) at (-1.5, 3) {}; 
  \node[vertex,fill=red!50] (n7) at (-1.0, 4) {}; 

  \draw[->, ultra thick ] (n3) to  [bend right=20] (n4) ;
  \draw[->, very thick ] (n4) -- (n5);
  \draw[->, thick ] (n4) to [bend right=20] (n7);

  \node[vertex,fill=orange!60] (n0) at (1, 0) {}; 
  \node[vertex,fill=orange!60] (n9) at (1.0, 3) {}; 
  \node[vertex,fill=orange!60] (n14) at (0, 5) {}; 
  \node[vertex,fill=orange!60] (n15) at (2, 6) {}; 

  \draw[->, ultra thick ] (n0) to [bend left=10] (n9);
  \draw[->, thick ] (n9) -- (n14);
  \draw[->, very thick ] (n9) to [bend right=10] (n15);
\end{tikzpicture}
\caption*{\bf C}
  \end{subfigure}
  \begin{subfigure}{0.24\textwidth}
  \begin{tikzpicture}
  [scale=0.7]
  \node[blank] (n3) at (-1, 0) {}; 
  \node[blank] (n0) at (1, 0) {}; 
  \node[blank] (n2) at (-0.5, 1) {}; 
  \node[blank] (n1) at (0.5, 1) {}; 

  \node[blank] (n4) at (-1, 2) {}; 
  \node[blank] (n5) at (-1.5, 3) {}; 
  \node[blank] (n6) at (-0.5, 3) {}; 
  \node[blank] (n7) at (-1.0, 4) {}; 

  \node[blank] (n8) at (1.0, 2) {}; 
  \node[blank] (n9) at (1.0, 3) {}; 
  \node[blank] (n10) at (0.5, 4) {}; 
  \node[blank] (n11) at (1.5, 4) {}; 
  \node[blank] (n12) at (2.0, 5) {}; 
  \node[blank] (n13) at (0.5, 2.5) {}; 
  \node[blank] (n14) at (0, 5) {}; 
  \node[blank] (n15) at (2, 6) {}; 

  \draw[ ultra thick ] (n3) -- (n2);% node[right]{$ \bf x$};
  \draw[ ultra thick ] (n2) -- (n1);
  \draw[ ultra thick ] (n0) -- (n1);

  \draw[ very thick ] (n2) -- (n4);
  \draw[ very thick ] (n4) -- (n5);
  \draw[ thick ] (n4) -- (n6);
  \draw[ thick ] (n6) -- (n7);

  \draw[ very thick ] (n1) -- (n8);
  \draw[ very thick ] (n8) -- (n9);

  \draw[ thick ] (n9) -- (n10);
  \draw[ thick ] (n10) -- (n14);

  \draw[ thick] (n9) -- (n13);

  \draw[ very thick ] (n9) -- (n11);
  \draw[ very thick ] (n11) -- (n12);
  \draw[ very thick ] (n12) -- (n15);

\end{tikzpicture}
\caption*{\bf D}
  \end{subfigure}
  \caption{Schematic illustrating network representations for the
    estimation of perivascular flow induced by vascular wall
    motion. (\textbf{A}) For a vascular network with vessels
    represented by edges of varying radii and length and connected at
    nodes with one or more (here two, marked in red and orange) supply
    nodes, we compute one subnetwork for each supply node by proximity
    (\textbf{B}). Each subnetwork is reduced to a minimal, bifurcating
    and directed tree (\textbf{C}) which is then used to compute
    $\langle Q' \rangle$.}
  \end{figure}
%\end{document}
