\subsection*{Intracranial solute influx and clearance during sleep}
\begin{itemize}
\item
  Sleep is known to affect a number of the transport characteristics including: increased extracellular volume fraction, altered vascular pulsatility and a-forteriori perivascular flow and mixing, reduced CSF production, reduced glymphatic transport.
\end{itemize}

\subsection*{Clearance and the brain's waterscape}
\begin{itemize}
\item 
  Modelling metabolite clearance rather than solute influx (Model
  C). Consider several diffusion coefficients corresponding to
  gadubutrol, dextran, tau and amyloid-beta, otherwise identical flow
  set-ups.
\end{itemize}


\subsection*{Influx and clearance in pathologies}
\begin{itemize}
\item
  A number of neurological and neurodegenerative disorders yield mechanical changes in the intracranial environment (increased arterial stiffness, cerebral arterial angiopathies, perivascular flow changes due to hypertension, altered CSF flow patterns, altered diffusion properties in cancer, BBB leakage). One key clinical aspect is increased PVS width in e.g. iNPH. 
\end{itemize}

\subsection*{Effect of BBB}

Leakage to the blood, some sink/non-zero permeability for the PVS-blood (BBB) interface (Model N).

