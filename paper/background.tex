\section{Background}

\mer{Point from Miro: Goriely and scaling from mice to men.}

\begin{enumerate}
\item
  Communication between perivascular spaces and the subarachnoid space
  \begin{itemize}
  \item
    Rennels et al~\cite{rennels1985evidence} infused tracer into the lateral cerebral ventricles and subarachnoid space of cats (and one dog) and observed that the tracer distributed in a perivascular pattern. The tracer distributed more rapidly around arterioles than around capillaries and veins. Importantly, the rapid paravascular influx was modulated by arterial pulsatility and in particular reduced by reduced vessel pulsatility. They directly postulate that ``fluid circulation through the central nervous system occurs via paravascular pathways'', and discuss both mixing and pumping as pulsatility-associated mechanisms for transport. 
  \item
    Weller and coauthors conducted pioneering studies of perivascular spaces, flow and transport~\cite{zhang1990interrelationships, zhang1992directional, weller2005microscopic}. They identified thin sheaths of pial cells surrounding arteries and arterioles (but not veins or venules) on the brain surface and within the human brain itself~\cite{zhang1990interrelationships}. Further, they observed that tracers spread along perivascular (arterial, venous and capillar) spaces in (rat) grey matter, and in the subarachnoid space (to the cribriform plate and nasal lymphatics).   
  \item
    Ichimura et al~\cite{ichimura1991distribution} study the distribution of large molecular weight tracers in the subarachnoid and surface perivascular spaces and within the cortex in rats. They observed rapid spread in perivascular spaces, and that tracers injected into the CSF of the subarachnoid space spread also into subarachnoid and cortical perivascular spaces. In conclusion, they highlight the presence of an extensive perivascular network, and report of some bulk flow, but slow and of variable directionality - in contrast to e.g Rennels et al~\cite{rennels1985evidence}. 
  \item In a series of papers, Bakker and coauthors study perivascular anatomy and solute transport~\cite{bedussi2017paravascular, bedussi2018paravascular}. They find that the subarachnoid space, the cisterns, ventricles and penetrating periarteriolar spaces form a continuous cerebrospinal fluid-filled space surrounding and penetrating into the murine brain \cite{bedussi2017paravascular}. Moreover, they demonstrate pulsatile and directional (antegrade) flow in perivascular (predominantly periarterial) spaces on the brain surface \cite{bedussi2018paravascular}. 
  \item
    Iliff et al~\cite{iliff2012paravascular} observed rapid movement of tracer (injected in the cisterna magna) along the outside of surface arteries and penetrating arterioles in mice, and state this is in agreement with the presence of perivascular sheaths as reviewed by Weller~\cite{weller2005microscopic}.
  \item
    The architecture of perivascular spaces in and around the rat brain was also studied in detail by Thorne and colleagues~\cite{pizzo2018intrathecal, hannocks2018molecular}. Hannocks et al~\cite{hannocks2018molecular} characterize the molecular composition of different perivascular compartments. Pizzo et al~\cite{pizzo2018intrathecal} emphasizes the presence of stomata or pores present on the interface between blood vessels and the CSF in the subarachnoid space. Neither seem to quantitatively report on PVS widths.
  \item Mestre et al~\cite{mestre2022periarteriolar} study the properties of pial perivascular spaces in detail in mice, and in particular the structure and coverage of pial cells and pial layers surrounding leptomeningeal arteries (and arterioles). They report of pial cells forming sheaths for larger ($> 10000 \mu$m$^2$ cross-section lumen area) surface arteries and partially coverage for smaller surface arteries, with higher coverage in ventral SAS regions. They find that these pial layers do not form an impermeable barrier to small molecules. Importantly, they also study the pial coverage in aged and Alzheimer's model mice, and find significant and complex changes in these pial structures.
  \item
    In humans, Eide and Ringstad investigate cerebrospinal fluid tracer transport in the human subarachnoid space after intrathecal injection~\cite{eide2024functional}. They observe tracer enrichment antegrade along the major cerebral arteries, and enrichment of the nearly cerebral cortex. They also consider variations in different patient cohorts and find impaired transport in iNPH patients with increased perivascular space size. They report of tracer enrichment in donut-shaped perivascular spaces. The timings here may be the best estimate of human PVS flow speeds. 
  \item
    Yamamoto et al~\cite{yamamoto2024perivascular} also report of contrast-enhancement in human PVS.
  \end{itemize}
\item
  Shapes and sizes of the perivascular spaces\footnote{The area $A$ of
  a circle with inner radius $R_1$ is $A = \pi R_1^2$. The area $A_2$
  of an annulus with inner radius $R_1$ and outer radius $R_2$ is $A_2
  = \pi (R_2^2 - R_1^2)$. The width of this annulus is $R_2 -
  R_1$. The ratio PVS/lumen area can thus be estimated as $A_2/A =
  (R_2^2 - R_1^2)/R_1^2$. For example, if $R_2 = n R_1$ ($n = 2
  \rightarrow$ PVS width equal to lumen radius, $n = 3 \rightarrow$
  PVS width equal to lumen diameter), then $A_2/A = n^2 - 1$. However,
  if the annulus has an elliptic outer boundary with radii $R_a$ and
  $R_b$, then its area is $A_e = \pi (R_a R_b - R_1^2)$, and so in the
  extreme case where $R_a > R_b = R_1$, the area is $\pi R_1 (R_a -
  R_1)$, and the PVS/lumen area ratio is $A_e/A = (R_a -
  R_1)/R_1$. And so, for $R_a = n R_1$, then $A_e/A = n-1$.}. Note
  that representing the surface PVS as annular cylinders is a
  simplification, and several studies emphasize the asymmetric nature
  of the PVS shape~\cite{mestre2018flow, tithof2019hydraulic,
    vinje2021brain, raicevic2023sizes}. Moreover, do not forget that
  the sizes of mice and humans are orders of magnitude apart.
  \begin{itemize}
  \item
    Ichimura et al~\cite{ichimura1991distribution} report on a typical
    perivascular width of $1-10\mu$m, but broader around vessel
    bifurcations in the rat subarachnoid space (fixed specimens).
  \item Foley et al~\cite{foley2012realtime} report on perivascular
    space width within rat cortex of $8-10\mu$m for arterioles of
    $\approx$30$\mu$m in diameter.
  \item
    Schain et al~\cite{schain2017cortical} report on the sizes of pial
    arteries ($6.0-11 \mu$m in diameter\footnote{Schain et
    al~\cite{schain2017cortical} state diameter rather, but the
    diameter does not match with the given area, so perhaps this
    should be the radius}) and veins ($9.4-21\mu$m) in mice, with an
    average PVS/lumen area ratio of $1.26$ for arteries and $0.13$ for
    veins. 
  \item Mestre et al~\cite{mestre2018flow} measure and characterize
    pial perivascular transport and estimate flow magnitudes in
    mice. They report a PVS/lumen area ratio of $1.4 \pm 0.1$ and a
    PVS width of $\approx 40 \mu$m which described as comparable to
    the adjacent artery (diameter). 
\item
  Raicevic et al~\cite{raicevic2023sizes} study the sizes and shapes
  of perivascular spaces surrounding pial arteries in mice in detail,
  in particular the relationship between periarterial space and lumen
  (cross-section) area and its variation with vessel area and
  location. They remark that the variation in PVS area is larger
  between PVS segments than along a single PVS segment. The PVS area
  seems to increase with the vessel area, but an affine approximation
  gives a poor fit, and from inspecting the distribution, the
  relationship seems superlinear. The peak distribution value of
  segmented PVS/lumen area is $1.12$.
  \item
    In terms of size, Bedussi et al.~\cite{bedussi2018paravascular} report a PVS width of $\approx$20$\mu$m surrounding surface arteries branching from the MCA (of diameters $45 \pm 7$ $\mu$m). Note that they include both data from mice and OCT data from humans. They report of human PVS size linearly related with the vessel size (and in the range 0.1--0.35 mm for vessels of diameter 0.1--0.5 mm).
  \item Vinje et al~\cite{vinje2021brain} and references therein illustrates the location and characteristics of human pial perivascular spaces, using human OCT data from the Bakker lab.
  \item Bollman et al~\cite{bollmann2022imaging} image the human pial vasculature at high resolution (7T), and report of pial arterial diameters in the range $50-280 \mu$m.
\item
  Smets et al~\cite{smets2024perivascular} study the size of
  perivascular spaces surrounding pial arteries and veins in mice
  (in-vivo), with particular emphasis on the properties of perivenous
  spaces. They observe the PVS as two ``triangular spaces'' adjacent
  to both arteries and veins, and separated from the subarachnoid
  space by a membrane (see also references
  to~\cite{zhang1990interrelationships, pizzo2018intrathecal,
    mollgard2023mesothelium}). They found that the cross-section area
  of the PVS correlates with the lumen area both for pial arteries
  and veins, but not for penetrating vessels. They report a
  PVS-to-lumen-area ratio of 0.43 for arteries and 0.35 for veins.
\item
  Personal communication (Ringstad, Oct 15 2024)~\cite{eide2024functional}: a typical blood vessel in the M2-segment of the MCA is 1.56 mm in diameter ($2 R_1$) with an annular PVS of width ($R_2 - R_1$) 1.23 mm and total PVS diameter $2 R_2$ of 3.93. In iNPH patients, extreme outliers are observed; e.g. still with annular PVS, but not concentric (shifted vessel), for vessel diameter of 2 mm and PVS width from 1.1 mm to 6.0 mm.
  \end{itemize}
\item
  Perivascular hydraulic resistances. The hydraulic resistance of
  perivascular spaces has been studied extensively, with key
  contributions from Thomas, Kelley, Tithof and collaborators~\cite{tithof2019hydraulic}.
  \begin{itemize}
  \item Boster et al.~\cite{boster2024hydraulic} study methods of estimating the hydraulic resistance of perivascular spaces via in-vivo imaging-based 3D reconstructions of pial perivascular spaces in mice; they report of hydraulic resistances of the order $1.7 \times 10^6 - 8.9 \times 10^5$ mmHg·min/mL/m).
  \end{itemize}
\item
  Perivascular flow characteristics: magnitudes and directionality?
  \begin{itemize}
  \item
    Rennels et al~\cite{rennels1985evidence} highlight in their
    Discussion that
    \begin{quote}
      This evidence of fluid displacement in both directions through
      the perivascular spaces suggest either that influx and efflux
      occur intermittently through the same perivascular spaces or
      that these fluid movements occur via separate populations of
      perivascular spaces.
    \end{quote}
  \item
    Ichimura et al~\cite{ichimura1991distribution} found that albumin in the subarachnoid periarterial space moved 1.5 mm away from the injection site in about 7 minutes, thus a net speed of around $3.6 \mu$m/s. 
  \item
    Hadaczek et al~\cite{hadaczek2006perivascular} studied the role of arterial pulsations in more detail, following up on~\cite{rennels1985evidence}.
  \item Foley et al~\cite{foley2012realtime} study perivascular transport characteristics of nanoparticles injected in the cerebral cortex of rats, though in the context of convection-enhanced delivery.
  \item
    In terms of movement, Bedussi et al~\cite{bedussi2018paravascular} found a net movement of particles in surface PVS (in mice) in the antegrade (same direction as blood flow) direction with an average velocity of $17 \pm 2\mu$m/s, and a mean amplitude of the pulsatile motion of $14 \pm 2 \mu$m.
\item Mestre et al~\cite{mestre2018flow} report a mean (averaged over time and space) flow speed of $18.7 \mu$m/s (root-mean-square velocity?) with net transport in the antegrade direction (near the MCA, in mice).
\item
  Asgari et al~\cite{asgari2016glymphatic} model tracer transport in axial periarterial spaces, highlight the negligible net (directional, bulk) flow induced by arterial wall pulsations at physiologically realistic wavelengths, but also the contributions from wall pulsations to dispersion. 
\item
  Rey and Sartinoranont~\cite{rey2018pulsatile} examine pulsatility as a driver for net flow in perivascular spaces within the brain parenchyma via a hydraulic network models, again highlighting the negligible net flow induced by vascular wall movements, but significant pulsatile flow and thus potential for dispersive effects~\cite{watson1983diffusion, asgari2016glymphatic}.
\item
  Rey et al~\cite{rey2023perivascular} consider a comprehensive ex-vivo perivascular network segmentation in the rat brain following in-vivo intraventricular contrast infusion. They highlight the numerous connections between the ventricular system and the perivascular spaces, including long perivascular segments extending from the surface and deep into the brain. 
\item
  In the context of dispersion, see also~\cite{asgari2016glymphatic, keith2019dispersion, troyetsky2021dispersion}
\item
  Wright et al~\cite{wright2024coupled} study the dynamics (waveforms over time) of human arterial blood flow (with resting-state fMRI) and perivascular CSF flow (with dynDWI, apparent diffusion coefficient (mm$^2$/s)). The waveforms seem coupled, with the arterial peak preceding the CSF peak by around 0.06 (of a cardiac cycle). The report ADC in the PVS on the order $2.5-5.5 \times 10^{-3}$ mm$^2$/s, with alterations in the coupling with age.
\item
  Boster and coauthors~\cite{boster2023artificial, toscano2024infeering} estimate perivascular flow parameters in mice via physics-informed neural networks (e.g.~pressure gradients, hydraulic resistances, shear stresses). Boster et al~\cite{boster2023artificial} estimate PVS pressure gradients on the order of $0.7-5 \times 10^{-4} \unit{Pa/\mu m}$\footnote{Note that 
  \begin{equation}
    1 \times 10^{-4} \, \unit{Pa/\mu m} = 1 \times 10^{2} \, \unit{Pa/m} \approx 100/133 \unit{mmHg/m} \approx 0.75 \unit{mmHg/m}
  \end{equation}
   }, downstream velocity component of $12.75 \pm 6.25 \unit{\mu m/s}$. 
  \end{itemize}
\item
  Broggini et al~\cite{broggini2024long} describe long-wavelength traveling waves of vasomotion (0.1 Hz) in pial arterioles in mice. They find waves traveling in both directions, with waves exiting the surface of the cortex more often than chance (85 of 140 exit). ``The wavelength of the observed pial waves is long, typically 20 mm or twice the length of the mouse cortex (inset in Figure 4C).'', with relative amplitudes in the range 0.02--0.12 in pial arterioles.
\item
  Kedarasetti et al~\cite{kedarasetti2020functional} model functional hyperemia waves (aka stimulus-induced vasomotion, see also van Veluw et al~\cite{vanveluw2020vasomotion}). Note that the directionality of vasomotion is classically thought to be in the opposite direction of the blood flow.
\item
  Note that Gokina et al~\cite{gokina1996electrical} report of spontaneous rhythmic contraction in human pial arteries.
\end{enumerate}
Cerebrospinal fluid flow
\begin{itemize}
\item
  Sweetman and Linninger~\cite{sweetman2011cerebrospinal} model the flow of cerebrospinal fluid in the cranial and spinal subarachnoid space of a healthy human driven by pulsatile displacement of the upper walls of the lateral ventricles representing volumetric changes of the vasculature, with model predictions in good agreement with clinical observations of CSF velocity magnitude and stroke volume. 
\item Causemann et al.~\cite{causemann2022human}
\item
  Vinje et al.~\cite{vinje2019respiratory} consider clinical measurements of intracranial pressure differences and human CSF-space geometries together with computational modelling to study CSF flow patterns in the aqueduct. They estimate the pressure gradient across the cerebral aqueduct resulting from a CSF production of 0.5L/day to be $0.009 \pm 0.006$ mmHg/m.
\item Hornkjøl et al~\cite{hornkjol2022csf} model CSF flow and solute transport in the human SAS and brain parenchyma. 
\end{itemize}
